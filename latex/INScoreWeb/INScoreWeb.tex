
\documentclass[a4paper,twoside]{article}
\usepackage{../INScore}


\makeatletter

\newcommand{\toplevel}[1]	{\section{#1}}
\newcommand{\sublevel}[1]	{\subsection{#1}}
\newcommand{\subsublevel}[1]	{\subsubsection{#1}}

\makeatother
\pagestyle{empty}


\begin{document}
\title{\vspace*{8cm}INScore Web \\ Version \inscoreversion}
\author{D. Fober\\ 
\\
\\
\\
\\
\\
\\
\\
\\
\\
\\
\\
\\
\\
\\
\\
\\
\\
\includegraphics[width=30mm]{../imgs/Logo_Grame}\\
Centre national de création musicale\\
}
\date{}


\maketitle
\thispagestyle{empty}
 
\cleardoublepage
\tableofcontents
\newpage
\setcounter{page}{1}
\pagestyle{plain}


%===============================
%:Introduction
\toplevel{Introduction}
\label{introduction}

Since version 1.27, the INScore engine is available as Javascript libraries:
\begin{itemize}
\item a WebAssembly [WASM] library providing all the services of the abstract INScore model,
\item a Javascript library providing an HTML view of the INScore model.
\end{itemize}

The web environment provides a very different runtime context than a native application: 
it is much more modular; due to the absence of a 'concrete machine' a number of INScore primitives do not make sense in a web environment; finally it provides new rendering capabilities with CSS. 

This document is intended to present the differences between the native and web versions of INScore. 
A special section is also devoted to the implementation of INScore Web in standalone HTML pages.

%===============================
%:Unsupported
\toplevel{Unsupported messages}
\label{unsupportedMsg}

%===============================
%:Unsupported
\toplevel{Unsupported attributes}
\label{unsupportedAttr}



\end{document}

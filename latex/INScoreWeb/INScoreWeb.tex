
\documentclass[a4paper,twoside]{article}
\usepackage{../INScore}


\makeatletter

\newcommand{\toplevel}[1]	{\section{#1}}
\newcommand{\sublevel}[1]	{\subsection{#1}}
\newcommand{\subsublevel}[1]	{\subsubsection{#1}}

\makeatother
\pagestyle{empty}


\begin{document}
\title{\vspace*{8cm}INScore Web \\ {\large Version \inscoreversion}}
\author{D. Fober\\ 
\\
\\
\\
\\
\\
\\
\\
\\
\\
\\
\\
\\
\\
\\
\\
\\
\\
\includegraphics[width=30mm]{../imgs/Logo_Grame}\\
Centre national de création musicale\\
}
\date{}


\maketitle
\thispagestyle{empty}
 
\cleardoublepage
\tableofcontents
\newpage
\setcounter{page}{1}
\pagestyle{plain}


%===============================
%:Introduction
\toplevel{Introduction}
\label{introduction}

Since version 1.27, the INScore engine is available as Javascript libraries:
\begin{itemize}
\item a WebAssembly [WASM] library providing all the services of the abstract INScore model,
\item a Javascript library providing an HTML view of the INScore model.
\end{itemize}

The web environment provides a very different runtime context than a native application: 
it is much more modular; due to the absence of a 'concrete machine' a number of INScore primitives do not make sense in a web environment; finally it provides new rendering capabilities with CSS. 

This document is intended to present the differences between the native and web versions of INScore. 
A special section is also devoted to the implementation of INScore Web in standalone HTML pages.

%===============================
%:Behavioural differences
\toplevel{Behavioural differences}
\label{behavior}

The OSC protocol is not supported in the Web version. As a result, the mode of communication with the INScore engine is different (see section \ref{communication}) and may also depend on the application that uses this engine. 

By default:
\begin{itemize}
\item the OSC output and error ports are redirected to the Javascript console.
\item drag \& drop works like in the native version: you can drop files or text to an INScore scene (an HTML div in the Web version). 
\end{itemize}

The \OSC{log} window (address \OSC{/ITL/log}) is dependent on the host application. By default, input messages addressed to the \OSC{log} node are directed to the Javascript console.

\sublevel{Optional components}
\label{components}

As the architecture of the web version is completely modular, the available objects depend on the host application:
e.g. a page which wants to use objects in symbolic notation (type \OSC{gmn}) will have to include the Guido library. 
This architecture allows applications to be optimized to fit their needs. It also facilitates the extension of the INScore engine.
The table \ref{componentTbl} presents the current supported components.

\begin{table}[H]
\begin{center}
\begin{tabular}{|c|l|l|}
\hline
Component & Name & Dependent types \\
\hline
Guido Engine & libGUIDOEngine.js & \OSC{gmn}, \OSC{gmnf}, \OSC{gmnstream}, \\
  &   & \OSC{pianoroll}, \OSC{pianorollf}, \OSC{pianorollstream} \\
MusicXML library & libmusicxml.js & \OSC{musicxml}, \OSC{musicxmlf} \\
  &   & use of MusicXML implies to have also the Guido Engine \\
Faust compiler & libfaust-wasm.js & \OSC{faust}, \OSC{faustf} \\
  & FaustLibrary.js &  \\
\hline
\end{tabular}
\end{center}
\caption{Components required by specific objects}
\label{componentTbl}
\end{table}%


\sublevel{Using files in a script}
\label{files}

You can use file based objects in an INScore script but the file path is interpreted differently: 
\begin{itemize}
\item when using an absolute path, it refers to the document root of the HTTP server,
\item when using a relative path, it refers to the location of the HTML page 
\end{itemize}

\note{}\\
Browsers infer a MIME type from the file extension and generally, download any file which extension is not recognized
(this behavior depends on the browser you are using). It is therefore recommended to use a \OSC{.txt} extension for any textual resource with non-standard extension. For example, a \OSC{score.gmn} file could be renamed and used as \OSC{score.gmn.txt}.


%===============================
%:Unsupported
\toplevel{Unsupported}
\label{unsupported}

%===============================
%: Unsupported types
\sublevel{Unsupported objects}
\label{unsupportedObjects}

The table \ref{unsupportedTbl} presents the objects that are not supported:

\begin{table}[H]
\begin{center}
\begin{tabular}{|c|l|}
\hline
Type & Comment \\
\hline
\OSC{fileWatcher} 	& doesn't make sense in a web environment \\
\OSC{httpd} 		& unsupported in a web environment  \\
\OSC{websocket} 	& unsupported in a web environment \\
Faust plugins 		& deprecated and redesigned (see section \ref{faustObjects}) \\
Gesture follower 	& unsupported \\
\hline
\end{tabular}
\end{center}
\caption{Unsupported objects}
\label{unsupportedTbl}
\end{table}%

The following components are not yet implemented:
\begin{itemize}
\item graphic signals objects: \OSC{graph}, \OSC{fastgraph}, \OSC{radialgraph}
\item Pianoroll stream: \OSC{pianorollstream} 
\item Misc.: \OSC{grid} \OSC{layer} 
\item Memory image: \OSC{memimg}
\item Sensors: \OSC{acceleromter}, \OSC{gyroscope}, \OSC{compass}, etc.
\end{itemize}


%===============================
%: Unsupported attributes
\sublevel{Unsupported messages}
\label{unsupportedMessages}

A number of messages are not supported in the web version, either because they do not make sense in the runtime context or because they cannot be implemented.

%--------------------------------
\subsublevel{Common messages}
\label{webcommonMessages}

\begin{itemize}
\item \OSC{export}, \OSC{exportAll}, \OSC{save}: are not supported due to the lack of file system
\item \OSC{shear}, \OSC{dshear}: not implemented
\item \OSC{effect}: \OSC{colorize}: not implemented
\item \OSC{edit}: not yet implemented
\end{itemize}

%--------------------------------
\subsublevel{Application messages}
\label{webappMessages}

\begin{itemize}
\item \OSC{quit}: do not make sense in the web environment
\item \OSC{rootPath}: do not make sense in the web environment
\item \OSC{mouse}: not yet implemented
\item \OSC{read}: not implemented
\item \OSC{load}: not implemented
\item \OSC{port}, \OSC{outport}, \OSC{errport}: do not make sense without OSC
\item \OSC{browse}: not implemented
\item \OSC{get}: \OSC{guido-version} \OSC{musicxml-version}: not implemented. \\
Note that the corresponding components are loaded, they print their version number to the Javascript console.
\end{itemize}

%--------------------------------
\subsublevel{Application log window}
\label{weblogMessages}

As mentioned above, the \OSC{log} window (address \OSC{/ITL/log}) is dependent on the host application. By default, input messages addressed to the \OSC{log} node are directed to the Javascript console.

\begin{itemize}
\item \OSC{clear}: dependent on the host application (do nothing by default)
\item \OSC{save}: not supported due to the lack of file system
\item \OSC{foreground}: dependent on the host application (do nothing by default)
\item \OSC{wrap}: dependent on the host application (do nothing by default)
\item \OSC{scale}: dependent on the host application (do nothing by default)
\item \OSC{write}: dependent on the host application (write to the Javascript console by default)
\end{itemize}


%--------------------------------
\subsublevel{Scene messages}
\label{websceneMessages}

\begin{itemize}
\item \OSC{new}, \OSC{del}: not implemented
\item \OSC{save}: not supported due to the lack of file system
\item \OSC{foreground}: dependent on the host application (do nothing by default)
\item \OSC{wrap}: dependent on the host application (do nothing by default)
\item \OSC{write}: dependent on the host application (write to the Javascript console by default)
\end{itemize}

%--------------------------------
\subsublevel{Type specific messages}
\label{webtypesMessages}

\begin{itemize}
\item \OSC{brushstyle}: not yet implemented
\end{itemize}

%--------------------------------
\subsublevel{Synchronization}
\label{webSynchronization}


%--------------------------------
\subsublevel{Events}
\label{webEvents}

touch events 
export
endPaint

%===============================
%:New messages
\toplevel{Specific new messages}
\label{newMessages}

class



%===============================
%:Faust objects
\toplevel{Faust objects}
\label{faustObjects}

Providing 


%===============================
%:Communication
\toplevel{Communication scheme}
\label{communication}




\end{document}

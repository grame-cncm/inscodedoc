
\documentclass[a4paper,twoside]{article}
\usepackage{../INScore}


\makeatletter

\newcommand{\toplevel}[1]	{\section{#1}}
\newcommand{\sublevel}[1]	{\subsection{#1}}
\newcommand{\subsublevel}[1]	{\subsubsection{#1}}

\newcommand{\inprogress}[1]	{{\color{blue} \texttt{(#1)}}}
\newcommand{\jsonindent}	{\hspace*{5mm}}

\makeatother
\pagestyle{empty}


\begin{document}
\title{\vspace*{8cm}INScore Web \\ {\large Version \inscoreversion}}
\author{D. Fober\\ 
\\
\\
\\
\\
\\
\\
\\
\\
\\
\\
\\
\\
\\
\\
\\
\\
\\
\includegraphics[width=30mm]{../imgs/Logo_Grame}\\
Centre national de création musicale\\
}
\date{}


\maketitle
\thispagestyle{empty}
 
\cleardoublepage
\tableofcontents
\newpage
\setcounter{page}{1}
\pagestyle{plain}


%===============================
%:Introduction
\toplevel{Introduction}
\label{introduction}

Since version 1.27, the INScore engine is available as Javascript libraries:
\begin{itemize}
\item a WebAssembly [WASM] library providing all the services of the abstract INScore model,
\item a Javascript library providing an HTML view of the INScore model.
\end{itemize}

The web environment provides a very different runtime context than a native application: 
it is much more modular; due to the absence of a 'concrete machine' a number of INScore primitives do not make sense in a web environment; finally it provides new rendering capabilities with CSS. 

This document is intended to present the differences between the native and web versions of INScore. 
A special section is also devoted to the implementation of INScore Web in standalone HTML pages.

%===============================
%:Main differences
\toplevel{Main differences}
\label{mainDiffs}

The OSC protocol is not supported in the Web version. As a result, the mode of communication with the INScore engine is different (see section \ref{communication}) and may also depend on the application that uses this engine. 

By default:
\begin{itemize}
\item the OSC output and error ports are redirected to the Javascript console.
\item drag \& drop works like in the native version: you can drop files or text to an INScore scene (an HTML div in the Web version). 
\end{itemize}

The \OSC{log} window (address \OSC{/ITL/log}) is dependent on the host application. By default, input messages addressed to the \OSC{log} node are directed to the Javascript console.

\sublevel{Optional components}
\label{components}

As the architecture of the web version is completely modular, the available objects depend on the host application:
e.g. a page which wants to use objects in symbolic notation (type \OSC{gmn}) will have to include the Guido library. 
This architecture allows applications to be optimized to fit their needs. It also facilitates the extension of the INScore engine.
The table \ref{componentTbl} presents the current supported components.

\begin{table}[H]
\begin{center}
\begin{tabular}{|c|l|l|}
\hline
Component & Name & Dependent types \\
\hline
Guido Engine & libGUIDOEngine.js & \OSC{gmn}, \OSC{gmnf}, \OSC{gmnstream}, \\
  &   & \OSC{pianoroll}, \OSC{pianorollf}, \OSC{pianorollstream} \\
MusicXML library & libmusicxml.js & \OSC{musicxml}, \OSC{musicxmlf} \\
  &   & use of MusicXML implies to have also the Guido Engine \\
Faust compiler & libfaust-wasm.js & \OSC{faust}, \OSC{faustf} \\
  & FaustLibrary.js &  \\
\hline
\end{tabular}
\end{center}
\caption{Components required by specific objects}
\label{componentTbl}
\end{table}%


\sublevel{Using files in a script}
\label{files}

You can use file based objects in an INScore script but the file path is interpreted differently: 
\begin{itemize}
\item when using an absolute path, it refers to the document root of the HTTP server,
\item when using a relative path, it refers to the location of the HTML page 
\end{itemize}

\note{}\\
Browsers infer a MIME type from the file extension and generally, download any file which extension is not recognized
(this behavior depends on the browser you are using). It is therefore recommended to use a \OSC{.txt} extension for any textual resource with non-standard extension. For example, a \OSC{score.gmn} file could be renamed and used as \OSC{score.gmn.txt}.


%===============================
%:Unsupported
\toplevel{Unsupported}
\label{unsupported}

%===============================
%: Unsupported objects
\sublevel{Unsupported objects}
\label{unsupportedObjects}

The table \ref{unsupportedTbl} presents the objects that are not supported:

\begin{table}[H]
\begin{center}
\begin{tabular}{|c|l|}
\hline
Type & Comment \\
\hline
\OSC{fileWatcher} 	& unsupported \\
\OSC{httpd} 		& unsupported server  \\
\OSC{websocket} 	& unsupported server \\
Faust plugins 		& deprecated and redesigned (see section \ref{faustObjects}) \\
Gesture follower 	& unsupported \\
\hline
\end{tabular}
\end{center}
\caption{Unsupported objects}
\label{unsupportedTbl}
\end{table}%

The following components are not yet implemented:
\begin{itemize}
\item graphic signals objects: \OSC{graph}, \OSC{fastgraph}, \OSC{radialgraph}
\item Pianoroll stream: \OSC{pianorollstream} 
\item Misc.: \OSC{grid} 
\item Memory image: \OSC{memimg}
\item Sensors: \OSC{acceleromter}, \OSC{gyroscope}, \OSC{compass}, etc.
\end{itemize}


%===============================
%: Unsupported messages
\sublevel{Unsupported messages}
\label{unsupportedMessages}

A number of messages are not supported in the web version, either because they do not make sense in the runtime context or because they cannot be implemented.

%--------------------------------
\subsublevel{Common messages}
\label{webcommonMessages}

\begin{itemize}
\item \OSC{export}, \OSC{exportAll}: not implemented
\item \OSC{shear}, \OSC{dshear}: not implemented
\item \OSC{effect}: \OSC{colorize}: not implemented
\item \OSC{edit}: not implemented
\end{itemize}

%--------------------------------
\subsublevel{Application messages}
\label{webappMessages}

\begin{itemize}
\item \OSC{quit}: do not make sense in the web environment
\item \OSC{rootPath}: not implemented
\item \OSC{mouse}: not yet implemented
\item \OSC{read}: not implemented
\item \OSC{port}, \OSC{outport}, \OSC{errport}: do not make sense without OSC
\end{itemize}

%--------------------------------
\subsublevel{Application log window}
\label{weblogMessages}

As mentioned above, the \OSC{log} window (address \OSC{/ITL/log}) is dependent on the host application. By default, input messages addressed to the \OSC{log} node are directed to the Javascript console.

\begin{itemize}
\item \OSC{clear}: dependent on the host application (do nothing by default)
\item \OSC{save}: not supported
\item \OSC{foreground}: dependent on the host application (do nothing by default)
\item \OSC{wrap}: dependent on the host application (do nothing by default)
\item \OSC{scale}: dependent on the host application (do nothing by default)
\item \OSC{write}: dependent on the host application (write to the Javascript console by default)
\end{itemize}


%--------------------------------
\subsublevel{Scene messages}
\label{websceneMessages}

\begin{itemize}
\item \OSC{foreground}: dependent on the host application (do nothing by default)
\item \OSC{frameless} \OSC{windowOpacity}: not supported
\end{itemize}

%--------------------------------
\subsublevel{Type specific messages}
\label{webtypesMessages}

\begin{itemize}
\item \OSC{brushstyle}: not yet implemented
\item Piano roll messages: not yet implemented
\item Video \OSC{get} messages: not yet implemented
\item SVG \OSC{animate}: not yet implemented
\item Arc \OSC{close}: not yet implemented
\item Line \OSC{arrows}: not yet implemented
\item \OSC{debug} node: not yet implemented
\end{itemize}

Symbolic score:
\begin{itemize}
\item \OSC{columns} \OSC{rows}: not supported
\item \OSC{pageFormat}: not yet implemented
\item \OSC{get}: \OSC{pageCount} \OSC{systemCount}: not yet implemented
\end{itemize}

%--------------------------------
\subsublevel{Synchronization}
\label{webSynchronization}

The following synchronisation modes are not yet supported:
\begin{itemize}
\item Stretch modes: \OSC{h}, \OSC{hv}
\end{itemize}

%--------------------------------
\subsublevel{Events}
\label{webEvents}

The following events are not yet supported:
\begin{itemize}
\item Touch events: \OSC{touchBegin}, \OSC{touchEnd}, \OSC{touchUpdate}
\item Url events: \OSC{success}, \OSC{error}, \OSC{cancel}
\item Score event: \OSC{pageCount}
\item \OSC{export}: not supported since the \OSC{export} message is not supported
\item \OSC{endPaint}: not supported
\end{itemize}

%===============================
%:Behavioral changes
\toplevel{Behavioral changes}
\label{behavior}

Some messages behave differently with the Web version.
\begin{itemize}
\item \OSC{save}: download the INScore content to the file given as argument. You should find this file in your download folder.
\end{itemize}


%===============================
%:New messages
\toplevel{Specific new messages}
\label{newMessages}

%--------------------------------
%:  CSS
\sublevel{Leveraging CSS}
\label{webCSS}

\href{https://www.w3schools.com/css/css_intro.asp}{Cascading Style Sheets} [CSS] are a powerful way to control the appearance of elements on a web page. The Web version of INScore provides a specific \OSC{class} message to use CSS in parallel to the standard mechanisms. This message is supported by all the INScore objects. 

\begin{rail}
classMsg : 'class' ([1]| [2] className +)
\end{rail}

\begin{itemize}
\item 1: without argument, remove all class settings
\item 2: set the CSS classes of an object
\end{itemize}

\note{}\\
If a \OSC{class} message has not the expected effect, it's likely because the CSS properties of the target object are set by the standard INScore mechanisms (like color, border, font-size, etc.). As the own attributes of an object have precedence over the class of the object, the properties of the class are then ignored. You can force the properties of the class by adding the CSS rule \OSC{!important} which will override all previous styling rules for that specific property.

%===============================
%:  Midi support
\toplevel{MIDI support}
\label{MIDI}

A \OSC{midi} node is automatically created at application level (\OSC{/ITL/midi}).
Before using the MIDI features, the \OSC{midi} node must be explicitly initialised using the \OSC{init} message. On success, the \OSC{ready} event is triggered, otherwise an \OSC{error} event is triggered.

\example
\sample{/ITL/midi watch ready ( do something to use the MIDI interface ); \\
/ITL/midi watch error ( do something else ); \\
/ITL/midi init;
}


%:  Midi events
\sublevel{MIDI events}
\label{MIDIEvents}


MIDI is supported using a specific event that you can configure using the \OSC{watch} message. Before using MIDI, you should check that your browser is supporting the Web MIDI API.

\begin{rail}
watchMIDI : 'watch' 'midi' midifilter
\end{rail}
A filter is used to select the MIDI messages that will trigger the event.

\begin{rail}
midifilter : (| '*')
			 | (| 'chan' int32) 
			 (midikey | 'prog' int32 | 'ctrl' int32 midivalue)
\end{rail}

\begin{itemize}
\item an empty filter can be used to stop watching MIDI input
\item '*' denotes any MIDI message (no filter at all)
\item \OSC{chan}: an optional channel number may be used to select only the MIDI messages on a given MIDI channel
\item midikey: is used to select key on/off messages according to their values and to an optional velocity.
\item \OSC{prog}: is used to select program change messages. A program change number is expected as next argument.
\item \OSC{ctrl}: is used to select control change messages. A control change number is expected as next argument.
\end{itemize}

\example\\
Accept all MIDI messages that are on channel 0
\sample{/ITL/scene/obj watch midi chan 0 (inscore messages list);
}

\begin{rail}
midikey : 
			 ('keyon' | 'keyoff') midivalue (| ('vel' midivalue))
\end{rail}

A MIDI key is either \OSC{keyon} or \OSC{keyoff}. It may be followed by an optional velocity selector.

A MIDI value is either literal values or a range of values.

\begin{rail}
midivalue : 
			(literal | range )
\end{rail}

\begin{rail}
literal : ([1] int32 | ([2] '[' (int32 +) ']' ))
\end{rail}

Literal values are:
\begin{itemize}
\item 1: a single value,
\item 2: a list of space separated values enclosed in brackets,
\end{itemize}
All the values must be in the MIDI data range (0-127).

\example\\
Accepting MIDI key on messages for 3 specific pitches.
\sample{/ITL/scene/obj watch midi keyon [60 62 64] (inscore messages list);
}

Range may be used when a value is within the specified range, or enters or leaves the range.

\begin{rail}
range : ([1] '(' int32 '-' int32 ')' )
		| [2] (')' int32 '-' int32 '(' )
		| [3] ('[' int32 '-' int32 ']' )
\end{rail}

\begin{itemize}
\item 1: trigger the event when the value enters the range. 
\item 2: trigger the event when the value leaves the range.
\item 3: trigger the event when the value is within the range.
\end{itemize}

\example\\
Accepting \OSC{keyon} MIDI messages only when entering and leaving the range 60 - 67;
\sample{/ITL/scene/obj watch midi keyon '[60-67]' (inscore messages list);\\
/ITL/scene/obj watch midi keyon ']60-67[' (inscore messages list);
}

\sublevel{MIDI verbose mode}
\label{verbMIDI}

In order to facilitate the detection of messages sent by a MIDI interface, a 'verbose' mode allows incoming messages to be displayed on the browser console. 
To enable or disable verbose mode, a specific message must be sent to the \OSC{/ITL/midi} object:

\begin{rail}
verbMIDI : 'verbose' int32
\end{rail}

The parameter is a numerical value that controls which MIDI events are displayed: 
\begin{itemize}
\item 0: disable the verbose mode
\item 1: displays the channel MIDI events (Real-time and system exclusive events are filtered out).
\item 2: displays all the MIDI events
\end{itemize}


\example
\sample{/ITL/midi verbose 1; 
}
Activates the MIDI verbose mode.


%===============================
%:Audio
\toplevel{Audio}
\label{audioObjects}

%===============================
%:Audio objects
\sublevel{Audio objects}
\label{audioObjects}

Audio objects are \emph{connectable} objects i.e. objects which output can be connected to the input of another audio object. In the INScore model, the following objects are audio objects: \OSC{audio}, \OSC{video}, \OSC{faust} (see section \ref{faustObjects}) and \OSC{audioio} (see section \ref{audioioObjects}).

Audio objects support the following messages:

\begin{rail}
audioMsgs : ('connect' | 'disconnect') ( ([1] destination 
			|  ( [2] int32 ':' destination ':' int32 )) + ) 
\end{rail}

\begin{itemize}
\item \OSC{connect}: connect the outputs of the object to the destination inputs. \OSC{destination} must be another audio object. Without  numeric prefix and suffix [1], the connexion may be subject of up or down mixing as specified by the \href{https://www.w3.org/TR/webaudio/#channel-up-mixing-and-down-mixing}{Web Audio API}. The destination supports regular expressions. The second form [2] connects an output channel to the a destination input channel. The first number refers to the target object channel, the second one to the destination channel. Channels are indexed from 1, using 0 is equivalent to all channels (that are then mixed together). See the examples below.
\item \OSC{disconnect}: disconnect the outputs of the object from the destination inputs. Note that errors (e.g. no existing connection with the destination) are silently ignored.
\end{itemize}

\example\\
Connect the first channel of an audio object to the second channel of the audio output :
\sample{/ITL/scene/obj connect '1:audioOutput:2';}
Connect all channels of an audio object to the first channel of the audio output :
\sample{/ITL/scene/obj connect '0:audioOutput:1';}

\note{}\\
The messages \OSC{/ITL/scene/obj connect '0:dest:0';} is equivalent to \OSC{/ITL/scene/obj connect dest;}

\note{}\\
The \OSC{connect} message assumes that the source and destination are located in the same hierarchy (i.e. they have the same parent). 

\begin{rail}
audiogetMsgs : 'get' ('connect' | 'in' | 'out')
\end{rail}

\begin{itemize}
\item \OSC{in}: gives the number of inputs of the audio object
\item \OSC{out}: gives the number of outputs of the audio object
\end{itemize}

%===============================
%:Faust objects
\sublevel{Faust objects}
\label{faustObjects}

\href{https://faust.grame.fr/}{Faust} is a functional programming language for sound synthesis and audio processing.
Faust objects are available, providing that the Faust library has been loaded.

\warning \\
A page containing Faust objects require an https connection, unless it runs on localhost.

%--------------------------------
\subsublevel{Creating a Faust object}
\label{webFaust}

The \OSC{faust} or \OSC{faustf} types must be used to create a Faust object.

\note{}\\
The \OSC{faust} type exists with the native version but to load a pre-compiled DSP. \OSC{faustdsp} and \OSC{faustdspf} types are not supported.

\begin{rail}
setFaust : 'set' (([1] 'faust' (| int32) dspCode)
				| ([2] 'faustf' (| int32) dspFile)
				| ([3] 'faustw' (| int32) wasmFile jsonFile))

\end{rail}

The expected arguments of the \OSC{set} message are:
\begin{itemize}
\item an optional integer that indicates a number of voices used to create a polyphonic DSP (see section \ref{webFaustPoly}). Note that when present, a polyphonic DSP is created even if equal to 1.
\item 1: Faust DSP code (see the \href{https://faustdoc.grame.fr/}{Faust language} for more information).
\item 2: a Faust DSP file.
\item 3: a Faust pre-compiled wasm file and the associated json file. Note that theses files can be generated using the \OSC{get wasm} message (see section \fullref{webFaustMsgs}). 
\end{itemize}

By default, a Faust DSP appears as a browsable block diagram.

\note{}\\
The Faust language uses characters that have a special meaning in HTML: for example, with the split operator \OSC{<:}, the '\OSC{<}' character will be interpreted as an opening HTML tag and you should use HTML escapes (e.g. \OSC{\&lt;} instead of \OSC{<}). This is necessary for inline DSP code only, DSP files are not concerned.

%--------------------------------
\subsublevel{Faust messages}
\label{webFaustMsgs}

Faust objects are active by default, i.e. the output signals is computed whatever the audio connection state. 
However, it is possible to disable the signals computation when it is not needed (e.g. for objects that are only temporarily active), which allows a large number of faust objects to be embedded in a score, while saving a significant CPU. The \OSC{compute} message is provided in this intend.

\begin{rail}
faustMsgs : 'compute' int32
\end{rail}

\begin{itemize}
\item \OSC{compute}: start or stop computing the audio signals. The parameter is a boolean value. Default value is 1.
\end{itemize}

Faust objects support the audio objects messages plus additional query messages:

\begin{rail}
faustgetMsgs : 'get' ('ui' | 'wasm')
\end{rail}

\begin{itemize}
\item \OSC{ui}: gives the Faust processor parameters (see section \ref{webFaustParams}) i.e. for each parameter: its OSC address followed by the parameter UI type, a label, the minimum and maximum values and the parameter step.
\item \OSC{wasm}: generates the Faust dsp as a precompiled wasm file with the associated json file. The Faust object name is used for the files name. On output, the files \OSC{xxx.wasm} and \OSC{xxx.json} should be present in the local download folder (where \OSC{xxx} is the Faust object name).
\end{itemize}


%--------------------------------
\subsublevel{Faust objects parameters}
\label{webFaustParams}

A Faust DSP code can declare UI elements that are used by \emph{architecture files} to build controllers providing users with dynamic control of the DSP parameters. In INScore, DSP UI elements are used to extend the Faust object address space. For example, when a DSP code declares a UI element named 'Volume', a Faust object which address is \OSC{/ITL/scene/dsp} is extended as \OSC{/ITL/scene/dsp/Volume}.

Faust parameters support two types of messages:

\begin{rail}
faustParamMsgs : [1] float32 | [2] 'get'
\end{rail}

\begin{itemize}
\item 1: set the parameter value
\item 2: gives the parameter value
\end{itemize}


%--------------------------------
\subsublevel{Polyphonic objects}
\label{webFaustPoly}

Polyphonic objects (i.e. Faust objects created using the optional voice number) support additional messages:

\begin{rail}
faustPolyMsgs : (('keyOn' | 'keyOff') chan pitch vel)
				| 'allNotesOff'
\end{rail}

\begin{itemize}
\item \OSC{keyOn} and \OSC{keyOff} messages take 3 integer parameters: a MIDI channel, the note pitch and the velocity.
\item \OSC{allNotesOff}: similar to MIDI all notes off message
\end{itemize}

%===============================
%:  Audio IO
\sublevel{Audio Input / Output}
\label{audioioObjects}

Audio input / output objects are provided as audio object (see section \ref{audioObjects}) to represent the physical audio inputs and outputs, in order to homogenize the connection process.
An audio input / output type is \OSC{audioio}. It is created with a number of inputs and outputs as arguments.

\begin{rail}
audioioMsgs : 'set' 'audioio' int32 int32
\end{rail}

\note\\
The current Javascript implementation automatically creates an audio input objet named \OSC{audioInput} (if any) and an audio output object named  \OSC{audioOutput} (if any). Thus the names \OSC{audioInput} and \OSC{audioOutput} are reserved and you should avoid to use them (unless audio connections are not needed).
 

%===============================
%:Communication
\toplevel{Communication scheme}
\label{communication}

INScore forwarding mechanism supports \OSC{http} and \OSC{ws} protocols since version 1.27. Since these protocols are connection based, a counterpart \OSC{connect} message is provided with the Web version.

\begin{rail}
connectMsg : ('connect' ( [1] | [2] remoteHost +))
\end{rail}

\begin{itemize}
\item 1) removes the existing set of connections,
\item 2) connect to a list of remote hosts. Once the connection is established, the local INScore engine may receive messages from the remote hosts. To establish the connection, the remote hosts must have set a forwarding mechanism using the same protocol with the same port number. 
\end{itemize}

\begin{rail}
remoteHost : (([1] 'http://' |  [2] 'https://' |  [3] 'ws://') hostname ':' portnum) +
\end{rail}

\begin{itemize}
\item 1) uses \OSC{http} as communication protocol,
\item 2) uses \OSC{https},
\item 3) uses \OSC{websockets}. 
\end{itemize}

\sublevel{Web messages format}
\label{msgFormat}

Messages emitted by the \OSC{http} and \OSC{ws} forwarding mechanism and received by connected clients are encoded in \href{https://www.json.org/}{JSON} and transmitted as base64 encoded packets. The JSON format is the following:
\sample{\{\\
\jsonindent 'id' : number , \\
\jsonindent 'method' : 'post' , \\
\jsonindent 'data' : textual inscore messages \\
 \}
}

\begin{itemize}
\item \OSC{id}: is a packet unique identifier (currently unused)
\item \OSC{method}: value must be 'post'
\item \OSC{data}: must contain a valid inscore script
\end{itemize}

Although the native version of INScore supports this format for the \OSC{http}, \OSC{https} and \OSC{ws} protocols, nothing prohibits another application to control INScore web pages, provided that the format described above is respected.



\end{document}

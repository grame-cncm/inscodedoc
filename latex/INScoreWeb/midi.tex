
--------------------------------
\sublevel{MIDI support}
\label{webMIDI}

\inprogress{this section is not yet implemented}

MIDI is supported using a specific event that you can configure using the \OSC{watch} message. 

\begin{rail}
watchMIDI : 'watch' 'midi' midifilter
\end{rail}
A filter is used to select the MIDI messages that will trigger the event.

\begin{rail}
midifilter : (| 'chan' int32) 
			 (| ('keyon' | 'keyoff' | 'vel' | 'prog' | 'ctrl') (literal | range ))
\end{rail}

\begin{itemize}
\item with an empty filter, any MIDI message will trigger the event
\item \OSC{chan}: an optional channel number may be used to select only the MIDI messages on a specific MIDI channel
\item \OSC{keyon}, \OSC{keyoff}, \OSC{vel}, \OSC{prog}, \OSC{ctrl}: are used to select MIDI messages according to their status and values.
\end{itemize}

\example\\
Filtering MIDI messages that are not on channel 0
\sample{/ITL/scene/obj watch midi chan 0 (inscore messages list);
}

Literal values are either a specific value or a list of values enclosed in brackets.

\begin{rail}
literal : (int32 | ('[' (int32 +) ']' ))
\end{rail}

\example\\
Accepting MIDI key on messages for 3 specific pitches.
\sample{/ITL/scene/obj watch midi keyon [60 62 64] (inscore messages list);
}


Range may be used when a value enters or leaves the specified range.

\begin{rail}
range : ([1] '[' int32 '-' int32 ']' )
		| [2] (']' int32 '-' int32 '[' )
\end{rail}

\begin{itemize}
\item 1: trigger the event when the value enters the range. 
\item 2: trigger the event when the value leaves the range.
\end{itemize}

\example\\
Accepting \OSC{keyon} MIDI messages only when entering and leaving the range 60 - 67;
\sample{/ITL/scene/obj watch midi keyon '[60-67]' (inscore messages list);\\
/ITL/scene/obj watch midi keyon ']60-67[' (inscore messages list);
}


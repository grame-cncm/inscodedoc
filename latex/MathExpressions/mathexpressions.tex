
% !TEX root = MathRef.tex

\newcommand{\mathstring}[1]		{\textit{string}(\OSC{#1})}
\newcommand{\mathstrnum}[1]		{\textit{length}(\OSC{#1})}
\newcommand{\mathsize}[1]		{\textit{size}(\OSC{#1})}
\newcommand{\op}				{$op$}
\newcommand{\todo}	[1]			{{\big [\texttt{\textbf{#1}}]}}

%===============================
%:Mathematical expressions
\toplevel{Mathematical expressions}
\label{mathExpr}

Since INScore version 1.20, mathematical expressions have been introduced as messages arguments. These expressions allows to compute values at parsing time.

%===============================
%:    Operators
\sublevel{Operators}
\label{Operators}

Basic mathematical expressions are supported. They are listed below.

\index{Math expression!operators}

\begin{rail}
mathexpr:     [1] matharg '+' matharg
			| [2] matharg '-' matharg
			| [3] '-' matharg
			| [4] matharg '*' matharg
			| [5] matharg '/' matharg
			| [6] matharg '\%' matharg
			| [7] '(' mathexpr ')'
			| [8] 'max' '(' (matharg +) ')'
			| [9] 'min' '(' (matharg +) ')'
			| [10]'(' mathbool '?' matharg ':' matharg ')'
\end{rail}

Note that \textit{matharg} could be a \textit{mathexpr} as well (see section \ref{mathargs}).

\begin{itemize}
\item \textbf{1}: addition.
\item \textbf{2}: substraction.
\item \textbf{3}: negation.
\item \textbf{4}: multiplication.
\item \textbf{5}: division.
\item \textbf{6}: modulo.
\item \textbf{7}: parenthesis, to be used for evaluation order.
\item \textbf{8}: gives the maximum value.
\item \textbf{9}: gives the minimum value.
\item \textbf{10}: conditional form: returns the first \OSC{matharg} if \OSC{mathbool} is true, otherwise returns the second one.
\end{itemize}

\example\\
Some simple mathematical expressions used as message parameters:
\sample{/ITL/scene/myObject x 0.5 * 0.2;\\
/ITL/scene/myObject y (\$var ? 1 : -1);
}

Boolean operations are the following:

\index{Math expression!booleans}

\begin{rail}
mathbool:     [1] arg
			| [2] '!' arg
			| [3] arg '==' arg
			| [4] arg '>' arg
			| [5] arg '>=' arg
			| [6] arg '<' arg
			| [7] arg '<=' arg
\end{rail}

\begin{itemize}
\item \textbf{1}: evaluate the argument as a boolean value.
\item \textbf{2}: evaluate the argument as a boolean value and negates the result.
\item \textbf{3}: check if the arguments are equal.
\item \textbf{4}: check if the first argument is greater than the second one.
\item \textbf{5}: check if the first argument is greater or equel to the second one.
\item \textbf{6}: check if the first argument is less than the second one.
\item \textbf{7}: check if the first argument is less or equal to the second one.
\end{itemize}

\example\\
Compare two variables:
\sample{/ITL/scene/myObject x (\$var1 > \$var2 ? 1 : -1);
}


%===============================
%:    Arguments
\sublevel{Arguments}
\label{mathargs}

Arguments of mathematical operations are the following:

\index{Math expression!arguments}

\begin{rail}
matharg:      [1] 'param'
			| [2] 'variable'
			| [3] 'variable' ('++' | '--')
			| [4] ('++' | '- -') 'variable' 
			| [5] 'msgvariable'
			| [6] mathexpr
\end{rail}

\begin{itemize}
\item \textbf{1}: any message parameter.
\item \textbf{2}: a variable value.
\item \textbf{3}: a variable that is post incremented or post decremented.
\item \textbf{4}: a variable that is pre incremented or pre decremented.
\item \textbf{5}: a message based variable.
\item \textbf{6}: a mathematical expression.
\end{itemize}



%===============================
%:    Polymorphism
\sublevel{Polymorphism}
\label{mathpoly}

Since INScore's parameters are polymorphic, the semantic of the operations are to be defined notably when applied to non numeric arguments. Actually, a basic OSC message parameter type is between \OSC{int32}, \OSC{float32} and \OSC{string}. However, due to the extension of the scripting language, parameters could also be arrays of any type, including mixed types (e.g. resulting from variable declarations). 

%===============================
\subsublevel{Numeric values}
\label{numbers}

For numeric arguments, automatic type conversion is applied with a precedence of \OSC{float32} i.e. when one of the argument's type is  \OSC{float32}, the result is also  \OSC{float32} (see Table \ref{table:mathopnum}).

\label{table:mathopnum}

\begin{table}[htbp]
  \centering
  \begin{tabular}{rlc}
    \hline
    arg1 & arg2 & output\\ 
    \hline
    \OSC{int32}		&  	\OSC{int32}		& \OSC{int32} \\ 
    \OSC{float32}	&  	\OSC{int32}		& \OSC{float32} \\ 
    \OSC{int32}		&  	\OSC{float32}	& \OSC{float32} \\ 
    \OSC{float32}	&  	\OSC{float32}	& \OSC{float32} \\ 
    \hline
  \end{tabular}
  \caption{Numeric operations output}
\end{table} 

%===============================
\subsublevel{Strings}
\label{strings}

For \OSC{string} parameters, operations that have an obvious semantic (like + applied to two strings) are defined (see Table \ref{table:mathopstr}) , the others are undefined and generate an error (see Table \ref{table:mathnoopstr}). 

\label{table:mathopstr}
\begin{table}[htbp]
  \centering
  \begin{tabular}{rlc}
    \hline
    operation & evaluates to & comment\\ 
    \hline
    \OSC{string} + \OSC{string} 		& \OSC{string} 						& concatenate the two strings \\ 
    \OSC{string} + \OSC{num} 			& \OSC{string} + \mathstring{num} 	& \OSC{num} is converted to string \\ 
    \OSC{num} + \OSC{string} 			& \mathstring{num} + \OSC{string} 	& " \\ 
    @max(\OSC{string} \OSC{string})		& \OSC{string} 						& select the largest string \\ 
    @min(\OSC{string} \OSC{string})		& \OSC{string} 						& select the smallest string \\ 
    \hline
  \end{tabular}
  \caption{Supported operations on strings}
\end{table} 

\label{table:mathnoopstr}

\begin{table}[htbp]
  \centering
  \begin{tabular}{rc}
    \hline
    operation & comment\\ 
    \hline
    \OSC{string} \op\ \OSC{string} 		& where \op\ is in [- * / \%]  \\ 
    \OSC{string} \op\ \OSC{num} 		& "  \\ 
    \OSC{num} \op\ \OSC{string} 		& "  \\ 
    -\OSC{string}  						&  \\ 
    prefixed or postfixed \OSC{string}  &  \\ 
    \hline
  \end{tabular}
  \caption{Non supported operations on strings}
\end{table} 

Boolean operations are supported on \OSC{string} only when both arguments are strings (see Table \ref{table:mathboolstr}). Other types combination generate an error.
	
\label{table:mathboolstr}

\begin{table}[htbp]
  \centering
  \begin{tabular}{rc}
    \hline
    boolean operation & evaluated as \\ 
    \hline
    \OSC{string} == \OSC{string} 		& regular strings comparison \\ 
    \OSC{string} > \OSC{string} 		& alphabetical strings comparison \\ 
    \OSC{string} >= \OSC{string} 		& " \\ 
    \OSC{string} < \OSC{string} 		& " \\ 
    \OSC{string} <= \OSC{string} 		& " \\ 
    \hline
  \end{tabular}
  \caption{Supported boolean operations on strings}
\end{table} 


%===============================
\subsublevel{Arrays}
\label{arrays}

For arrays, the operation is distributed inside the array elements:
\[
 arg\ op\ [v_1\ ...\ v_n]  := [arg\ op\ v_1\ ...\ arg\ op\ v_n] 
\]
or
\[ 
[v_1\ ...\ v_n]\ op\ arg  := [v_1\ op\ arg\ ...\ v_n\ op\ arg] 
\]

When both parameters are arrays, the operation is distributed from one array elements to the other array elements when the arrays have the same size and it generates an error when the sizes differ:
\[
[a_1\ ...\ a_i]\ op\ [b_1\ ...\ b_i]  := [a_1\ op\ b_1\ ...\ a_i\ op\ b_i\ ] 
\]

Boolean operations on arrays are evaluated as the logical $and$ of it's element's boolean values and generate an error when the arrays sizes differ.

